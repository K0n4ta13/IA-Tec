\documentclass[14pt]{extarticle}

\usepackage[spanish]{babel}
\usepackage[utf8]{inputenc}
\usepackage[letterpaper, margin=1in]{geometry}
\usepackage{graphicx}

\usepackage{csquotes}
\DeclareQuoteStyle{spanish}{\textquotedblleft}{\textquotedblright}{\textquoteleft}{\textquoteright}
\MakeOuterQuote{"}

\setlength{\parindent}{0pt}

\begin{document}
  
  \thispagestyle{empty}

  \includegraphics[width=\textwidth]{/usr/share/templates/latex/tec.jpeg}

  \begin{center}
    {\LARGE \textbf{Instituto Tecnológico de Culiacán}}

    \vspace{1cm}
    {\Large Inteligencia Artificial}

    \vspace{4cm}
    {\large \textbf{Equipo:}}

    \vspace{0.3cm}
    {\large
      Ramos Matunaga Raúl Alejandro\\\vspace{0.3cm}
      Uriarte Lopez Brandon Gael
    }

    \vspace{2cm}
    {\large \textbf{Carrera:}}

    \vspace{0.3cm}
    {\large Ingeniería en Sistemas Computacionales}

    \vspace{2cm}
    {\large \textbf{Docente:}}

    \vspace{0.3cm}
    {\large Mora Felix Zuriel Dathan}
  \end{center}

  \newpage

  {\Large \textbf{Historia de la IA}}

  \vspace{0.5cm}
  Warren McCulloch y Walter Pitts desarrollaron el primer modelo matemático de neuronas naturales dedicadas a la función cerebral humana como el comienzo de la IA en 1940. Más tarde, en la década de 1950, Alan Turing lanzó la "Prueba de Turing", un criterio para evaluar la capacidad de una máquina para exhibir un comportamiento inteligente similar al humano. Estos primeros desarrollos sentaron las bases para el estudio formal de la inteligencia artificial en 1951, Marvin Minsky y Dean Edmond, presentaron la primera computadora de red neuronal, SNARC. En 1952, Arthur Samuels de la IBM hace un programa que juega damas, hacia 1955 Samuels incorpora capacidades de aprendizaje al mismo programa.

  \vspace{0.3cm}
  En 1956 , se llevó a cabo la Conferencia de Verano en Inteligencia Artificial, esto constituyó la partida de nacimiento de la Inteligencia Artificial como campo de investigación, Además, los participantes en esta Conferencia sería considerados hasta ahora los más importantes investigadores en el campo, En 1969, Marvin Minsky y Seymour Papert publican Perceptrons, donde demuestran algunas limitaciones de las redes neuronales, generando, parcialmente, lo que se ha denominado ‘El Invierno de la Inteligencia Artificial’ un periodo de perdida de confianza y ausencia de fondos.

  \vspace{0.3cm}
  A pesar de este estancamiento, la investigación continuó, y en la década de 1980, los sistemas expertos adquirieron relevancia. Estos sistemas utilizaban reglas predefinidas para simular el razonamiento humano en áreas como la medicina y la ingeniería. Sin embargo, su alto costo y limitaciones en el aprendizaje autónomo provocaron otro período de declive, conocido como el "segundo invierno de la IA" en los años de 1987 a 1993.

  \vspace{0.3cm}
  El resurgimiento de la IA comenzó a finales del siglo XX con el desarrollo de nuevas técnicas de aprendizaje automático y el aumento en la capacidad de procesamiento de las computadoras. En 1997, la supercomputadora Deep Blue de IBM derrotó al campeón mundial de ajedrez Garry Kasparov, marcando un hito en la historia de la IA. Posteriormente, en 2006, Geoffrey Hinton y su equipo revitalizaron el interés en las redes neuronales profundas, lo que permitió grandes avances en el reconocimiento de voz, imágenes y procesamiento del lenguaje natural.

  \vspace{0.3cm}
  La última década ha sido testigo de un crecimiento exponencial en el campo de la IA, impulsado por el acceso masivo a datos y la mejora de los algoritmos de aprendizaje profundo. En 2012, AlexNet revolucionó la visión por computadora al ganar la competencia ImageNet, demostrando la eficacia del deep learning. En 2016, AlphaGo, desarrollado por DeepMind, venció al campeón mundial de Go, un juego mucho más complejo que el ajedrez, lo que evidenció el potencial de la IA para resolver problemas altamente sofisticados.

  \vspace{0.3cm}
  A partir de 2020, la inteligencia artificial generativa ha cobrado protagonismo con modelos como GPT-3 y GPT-4, que han transformado la forma en que los humanos interactúan con las máquinas. ChatGPT, basado en estos modelos, ha demostrado la capacidad de generar textos de manera coherente y relevante en múltiples contextos. Además, herramientas como DALL·E han permitido la creación de imágenes mediante descripciones en lenguaje natural, ampliando las aplicaciones de la IA en el ámbito creativo.
\end{document}
