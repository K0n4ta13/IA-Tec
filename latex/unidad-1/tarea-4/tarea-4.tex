\documentclass[14pt]{extarticle}

\usepackage[spanish]{babel}
\usepackage[utf8]{inputenc}
\usepackage[letterpaper, margin=1in]{geometry}
\usepackage{graphicx}

\usepackage[backend=biber]{biblatex} 
\addbibresource{bibliography.bib}

\usepackage{csquotes}
\DeclareQuoteStyle{spanish}{\textquotedblleft}{\textquotedblright}{\textquoteleft}{\textquoteright}
\MakeOuterQuote{"}

\setlength{\parindent}{0pt}

\begin{document}
  
  \thispagestyle{empty}

  \includegraphics[width=\textwidth]{/usr/share/templates/latex/tec.jpeg}

  \begin{center}
    {\LARGE \textbf{Instituto Tecnológico de Culiacán}}

    \vspace{1cm}
    {\Large Inteligencia Artificial}

    \vspace{4cm}
    {\large \textbf{Equipo:}}

    \vspace{0.3cm}
    {\large
      Ramos Matunaga Raúl Alejandro\\\vspace{0.3cm}
      Uriarte Lopez Brandon Gael
    }

    \vspace{2cm}
    {\large \textbf{Carrera:}}

    \vspace{0.3cm}
    {\large Ingeniería en Sistemas Computacionales}

    \vspace{2cm}
    {\large \textbf{Docente:}}

    \vspace{0.3cm}
    {\large Mora Felix Zuriel Dathan}
  \end{center}

  \newpage

  {\large \textbf{Agentes Deliberativos}}

  \vspace{0.5cm}
  Los \textbf{agentes deliberativos} son sistemas inteligentes autónomos que disponen de un modelo interno del mundo, lo que les permite razonar de forma simbólica y planificar sus acciones. A diferencia de los agentes reactivos, que responden de manera inmediata a estímulos sin un procesamiento interno profundo, los agentes deliberativos utilizan técnicas de razonamiento lógico y planificación para elegir la mejor secuencia de acciones a seguir. Entre sus características fundamentales destacan:

  \begin{itemize}
    \item \textbf{Representación Interna del Conocimiento:} Utilizan modelos simbólicos para representar las creencias (conocimiento sobre el entorno), los deseos (objetivos o metas a alcanzar) y las intenciones (planes seleccionados para cumplir dichos objetivos). Esta arquitectura se conoce como BDI (Beliefs, Desires, Intentions).
    \item \textbf{Ciclo de Percepción-Deliberación-Acción:} El agente primero percibe su entorno, actualiza su base de conocimiento y, a partir de ahí, delibera sobre qué objetivo perseguir y qué plan de acción ejecutar.
    \item \textbf{Planificación Estratégica y Razonamiento:} Gracias a su capacidad de razonamiento simbólico, estos agentes pueden evaluar múltiples opciones y predecir consecuencias a mediano y largo plazo.
    \item \textbf{Flexibilidad y Adaptación:} Pueden reestructurar sus planes en función de la evolución del entorno, lo que les permite alcanzar objetivos de forma más “inteligente” y adaptativa.
  \end{itemize}

  \vspace{1cm}
  {\large \textbf{Ejemplos de Empresas Reales que Utilizan Agentes Deliberativos}}

  \vspace{0.7cm}
  \textbf{IBM Watson}

  \vspace{0.3cm}
  IBM ha desarrollado sistemas de inteligencia artificial que incorporan componentes deliberativos para el análisis y toma de decisiones. Watson, por ejemplo, analiza grandes volúmenes de datos, actualiza sus conocimientos y genera planes de acción en aplicaciones como el diagnóstico médico y la atención al cliente.

  \vspace{1.1cm}
  \textbf{Amazon}

  \vspace{0.3cm}
  En logística y gestión de inventarios, Amazon utiliza agentes deliberativos para optimizar la cadena de suministro. Estos sistemas analizan datos en tiempo real (niveles de inventario, demanda del mercado, condiciones de tráfico) y planifican rutas de entrega y reposiciones de stock.

  \vspace{0.7cm}
  \textbf{Empresas de Telecomunicaciones (Claro, Tigo, Telintel)}

  \vspace{0.3cm}
  Varias compañías en América Latina han adoptado agentes deliberativos para mejorar la atención al cliente y la gestión de redes, optimizando el servicio y resolviendo incidencias en tiempo real.

  \vspace{0.7cm}
  \textbf{SAP y Oracle}

  \vspace{0.3cm}
  Las plataformas de gestión empresarial de SAP y Oracle incorporan agentes deliberativos que asisten en la planificación, asignación de tareas y optimización de recursos, mejorando la eficiencia organizativa.

  \vspace{0.7cm}
  \textbf{Waymo y Tesla}

  \vspace{0.3cm}
  En el sector automotriz, los vehículos autónomos de Waymo y Tesla integran componentes deliberativos para planificar rutas y tomar decisiones en situaciones imprevistas, combinando datos sensoriales con algoritmos de planificación.

  \newpage

  \thispagestyle{empty}

  \nocite{*}
  \printbibliography
  
\end{document}
