\documentclass[14pt]{extarticle}

\usepackage[spanish]{babel}
\usepackage[utf8]{inputenc}
\usepackage[letterpaper, margin=1in]{geometry}
\usepackage{graphicx}

\usepackage{csquotes}
\DeclareQuoteStyle{spanish}{\textquotedblleft}{\textquotedblright}{\textquoteleft}{\textquoteright}
\MakeOuterQuote{"}

\setlength{\parindent}{0pt}

\begin{document}
  
  \thispagestyle{empty}

  \includegraphics[width=\textwidth]{/usr/share/templates/latex/tec.jpeg}

  \begin{center}
    {\LARGE \textbf{Instituto Tecnológico de Culiacán}}

    \vspace{1cm}
    {\Large Inteligencia Artificial}

    \vspace{4cm}
    {\large \textbf{Equipo:}}

    \vspace{0.3cm}
    {\large
      Uriarte Lopez Brandon Gael\\\vspace{0.3cm}
      Ramos Matunaga Raúl Alejandro
    }

    \vspace{2cm}
    {\large \textbf{Carrera:}}

    \vspace{0.3cm}
    {\large Ingeniería en Sistemas Computacionales}

    \vspace{2cm}
    {\large \textbf{Docente:}}

    \vspace{0.3cm}
    {\large Mora Felix Zuriel Dathan}
  \end{center}

  \newpage

  \section*{Paradigma Simbólico}

  El enfoque simbólico, también conocido como \textbf{IA clásica}, se basa en la representación explícita del conocimiento mediante símbolos y reglas lógicas. Este paradigma asume que la inteligencia puede replicarse mediante la manipulación estructurada de símbolos abstractos.

  \subsection*{Características Principales}
  \begin{itemize}
      \item Representación del conocimiento mediante \textbf{reglas lógicas}
      \item Uso de \textbf{estructuras formales} como árboles de decisión y grafos conceptuales
      \item Énfasis en el \textbf{razonamiento deductivo} y la planificación
      \item Requiere \textbf{knowledge engineering} para codificar conocimiento experto
  \end{itemize}

  \subsection*{Ejemplos Prácticos}
  \begin{enumerate}
    \item \textbf{Sistemas Expertos Médicos}
    
      \textit{MYCIN (1976)}
    
      Sistema diagnóstico para enfermedades infecciosas que utilizaba 600 reglas clínicas. Analizaba síntomas mediante inferencia lógica para sugerir tratamientos con antibióticos.

    \item \textbf{Motores de Reglas Comerciales}
    
      \textit{IBM Operational Decision Manager}
      
      Plataforma empresarial que automatiza decisiones complejas usando reglas de negocio estructuradas (ej. aprobación de créditos basada en políticas financieras).

    \item \textbf{Procesamiento de Lenguaje Natural Simbólico}
    
      \textit{Gramáticas Formales en Traducción Automática}
      
      Sistemas como SYSTRAN (usado inicialmente por Google Translate) que aplicaban reglas gramaticales y diccionarios estructurados para traducción entre idiomas.

    \item \textbf{Planeamiento Automático}
    
      \textit{STRIPS (Stanford Research Institute Problem Solver)}

      Sistema pionero para control de robots que utilizaba lógica de primer orden para generar secuencias de acciones alcanzando objetivos específicos.

    \item \textbf{Grafos de Conocimiento}
    
      \textit{Knowledge Graph de Google}

      Base de conocimiento estructurado que relaciona entidades (personas, lugares, conceptos) mediante propiedades y relaciones semánticas formalizadas.
  \end{enumerate}

  \subsection*{Ventajas y Limitaciones}

  \textbf{Ventajas:}
  \begin{itemize}
    \item Transparencia en la toma de decisiones
    \item Facilidad para verificar el comportamiento
    \item Eficaz en dominios bien definidos
\end{itemize}

  \textbf{Limitaciones:}
  \begin{itemize}
    \item Requiere conocimiento experto previo
    \item Escalabilidad limitada
    \item Dificultad con datos ambiguos
  \end{itemize}

  \section*{Paradigma Conexionista}

  El enfoque conexionista, también llamado \textbf{aprendizaje profundo} o \textbf{basado en redes neuronales}, se centra en el aprendizaje automático mediante modelos inspirados en el cerebro biológico. Utiliza redes de nodos interconectados que ajustan sus pesos para detectar patrones en datos.

  \subsection*{Características Principales}
  \begin{itemize}
    \item Aprendizaje basado en \textbf{datos} en lugar de reglas explícitas
    \item Representación distribuida del conocimiento en pesos sinápticos
    \item Procesamiento \textbf{no lineal} y paralelo
    \item Capacidad de \textbf{generalización} a partir de ejemplos
    \item Arquitecturas jerárquicas con múltiples capas ocultas
  \end{itemize}

  \subsection*{Ejemplos Prácticos}
  \begin{enumerate}
    \item \textbf{Reconocimiento de Imágenes}
    
      \textit{ResNet (Microsoft, 2015)}
      
      Red neuronal convolucional (CNN) que superó a humanos en clasificación de ImageNet. Usada en diagnósticos médicos por imágenes y sistemas de seguridad facial.

    \item \textbf{Procesamiento de Lenguaje Natural (PLN)}
    
      \textit{BERT (Google, 2018)}

      Modelo de transformadores que entiende contexto bidireccional. Base de ChatGPT y sistemas de búsqueda semántica.

    \item \textbf{Reinforcement Learning}
    
      \textit{AlphaGo (DeepMind, 2016)}
    
      Red neuronal que derrotó al campeón mundial de Go mediante aprendizaje por refuerzo profundo (DQN).

    \item \textbf{Vehículos Autónomos}
    
      \textit{Tesla Autopilot}
    
      Sistema de percepción basado en redes neuronales para detección de peatones, señales de tráfico y toma de decisiones en tiempo real.

    \item \textbf{Recomendadores Inteligentes}
    
      \textit{Sistema de recomendación de Netflix}
      
      Modelos híbridos (redes neuronales + filtrado colaborativo) que predicen preferencias de usuarios con >75\% de precisión.
  \end{enumerate}

  \subsection*{Ventajas y Limitaciones}

  \textbf{Ventajas:}
  \begin{itemize}
    \item Manejo eficiente de datos no estructurados (imágenes, texto)
    \item Escalabilidad en problemas de alta dimensionalidad
    \item Adaptabilidad a nuevos patrones mediante entrenamiento
    \item Extracción automática de características relevantes
  \end{itemize}

  \textbf{Limitaciones:}
  \begin{itemize}
    \item Comportamiento de "caja negra"
    \item Requiere grandes volúmenes de datos etiquetados
    \item Coste computacional elevado para entrenamiento
    \item Riesgo de sobreajuste en datos sesgados
  \end{itemize}

  \textbf{Aplicación Actual}: Dominio en visión por computadora, traducción automática, generación de contenido y sistemas de predicción complejos.

  \section*{Paradigma Bioinspirado}

  Los sistemas bioinspirados imitan mecanismos de la naturaleza (evolución biológica, comportamientos colectivos o procesos fisiológicos) para resolver problemas complejos. Combina principios de biología, física y computación.

  \subsection*{Características Principales}
  \begin{itemize}
    \item Modelado de \textbf{procesos naturales}: evolución, enjambres, sistemas inmunológicos
    \item \textbf{Auto-organización} y emergencia de patrones globales desde reglas locales simples
    \item Adaptación dinámica mediante \textbf{retroalimentación} continua
    \item Tolerancia a fallos y \textbf{robustez} en entornos cambiantes
    \item Optimización colectiva sin control centralizado
  \end{itemize}

  \subsection*{Ejemplos Prácticos}
  \begin{enumerate}
    \item \textbf{Algoritmos Genéticos}
    
      \textit{Diseño de antenas para NASA}
    
      Evolución de estructuras mediante selección artificial (ST5-3-10: antena satelital con 70\% más eficiencia que diseños humanos).

    \item \textbf{Optimización por Colonias de Hormigas}
    
      \textit{Enrutamiento en redes de telecomunicaciones}
    
      Sistemas como AntNet ajustan rutas basándose en feromonas virtuales, reduciendo congestión en tiempo real.

    \item \textbf{Enjambres de Robots}
    
      \textit{Kilobots (Harvard)}
    
      1,024 robots simples que coordinan movimientos imitando cardúmenes, usados en agricultura de precisión.

    \item \textbf{Sistemas Inmunológicos Artificiales}
    
      \textit{Detección de intrusiones en ciberseguridad}
    
      AISEC emplea linfocitos virtuales para identificar patrones anómalos en redes corporativas.

    \item \textbf{Redes Neuronales Evolutivas}
    
      \textit{NEAT (NeuroEvolution of Augmenting Topologies)}
    
      Algoritmo que evoluciona estructura y pesos de redes neuronales simultáneamente, usado en control de drones.
  \end{enumerate}

  \subsection*{Ventajas y Limitaciones}

  \textbf{Ventajas:}
  \begin{itemize}
    \item Resuelve problemas NP-duros donde métodos clásicos fallan
    \item Escalabilidad en sistemas distribuidos
    \item Tolerancia a fallos parciales
    \item No requiere modelo matemático previo del problema
  \end{itemize}

  \textbf{Limitaciones:}
  \begin{itemize}
    \item Alto costo computacional en iteraciones
    \item Dificultad para garantizar convergencia óptima
    \item Sensibilidad a parámetros de configuración iniciales
    \item Complejidad en la validación teórica
  \end{itemize}

  \textbf{Aplicación Actual}: Optimización logística, robótica colaborativa, diseño de materiales, y gestión de recursos energéticos.

  \subsection*{Ejemplos Adicionales}

  \begin{enumerate}
    \item \textbf{Optimización de Parques Eólicos usando Enjambre de Partículas (PSO)}

    \begin{itemize}
      \item \textbf{Problema}: Disposición óptima de turbinas eólicas para minimizar interferencias aerodinámicas.
      
      \item \textbf{Aplicación del paradigma}:

        \begin{itemize}
          \item Inspiración en el comportamiento colectivo de bandadas de pájaros: cada partícula (solución candidata) ajusta su trayectoria combinando conocimiento individual y grupal.
          
          \item Mecanismo bioinspirado: Actualización de velocidades mediante ecuaciones que replican la coordinación descentralizada en la naturaleza.
        \end{itemize}
        
      \item \textbf{Beneficios del paradigma}:

        \begin{itemize}
          \item Capacidad para explorar espacios de solución no convexos sin requerir gradientes matemáticos.

          \item Adaptabilidad a cambios en las condiciones del viento mediante dinámica de enjambre autoorganizada.
        \end{itemize}
    \end{itemize}

    \item \textbf{Detección de Melanoma con Redes de Hongos}
    
      \begin{itemize}
        \item \textbf{Problema}: Segmentación de lesiones cutáneas en imágenes con ruido e irregularidades.
        
        \item \textbf{Aplicación del paradigma}:

          \begin{itemize}
            \item Modelado del crecimiento adaptativo de micelios fúngicos (Physarum polycephalum): las "hifas" virtuales exploran la imagen priorizando regiones de alto contraste.
            
            \item Mecanismo bioinspirado: Tácticas de exploración basadas en retroalimentación química simulada (atracción/repulsión a píxeles).
          \end{itemize}
          
        \item \textbf{Beneficios del paradigma}:
          \begin{itemize}
            \item Robustez ante artefactos en imágenes (sombras, vellos) gracias a la naturaleza probabilística de la exploración biológica.
            \item Detección de patrones morfológicos no lineales sin requerir dataset etiquetado extensivo.
          \end{itemize}
      \end{itemize}

    \item \textbf{Logística Urbana con Comportamiento de Bacterias}
    
      \begin{itemize}
        \item \textbf{Problema}: Planificación de rutas de entrega con restricciones dinámicas (tráfico, zonas de bajas emisiones).
        
        \item \textbf{Aplicación del paradigma}:

          \begin{itemize}
            \item Implementación del algoritmo de Forrajeo Bacteriano (BFO): las rutas se optimizan replicando la quimiotaxis de E. coli hacia nutrientes.
            
            \item Mecanismo bioinspirado: Eliminación de soluciones ineficientes mediante un proceso análogo a la muerte celular bacteriana.
          \end{itemize}
          
        \item \textbf{Beneficios del paradigma}:
        
          \begin{itemize}
            \item Adaptación en tiempo real mediante mecanismos de atracción/repulsión equivalentes a señales químicas.

            \item Manejo de múltiples objetivos contrapuestos (tiempo vs. emisiones) mediante estrategias de diversidad genética preservada.
          \end{itemize}
      \end{itemize}
  \end{enumerate}

  \section*{Paradigma Computacional}

  Este paradigma se fundamenta en modelos matemáticos y métodos algorítmicos para procesar información, enfatizando la eficiencia computacional y la resolución numérica de problemas. Integra teoría de la computación, estadística avanzada y optimización matemática.

  \subsection*{Características Principales}
  \begin{itemize}
    \item Basado en \textbf{modelos matemáticos formales} (ecuaciones, matrices, grafos)
    \item Énfasis en la \textbf{eficiencia algorítmica} (complejidad temporal/espacial)
    \item Uso intensivo de \textbf{métodos estadísticos} y análisis cuantitativo
    \item Integración con arquitecturas de \textbf{alto rendimiento} (HPC, GPUs)
    \item Enfoque \textbf{determinista} o probabilístico según el modelo
  \end{itemize}

  \subsection*{Ejemplos Prácticos}
  \begin{enumerate}
    \item \textbf{Algoritmos de Aprendizaje Automático}
    
      \textit{Máquinas de Soporte Vectorial (SVM)}
    
      Clasificador lineal/no lineal usado en diagnóstico genético (ej. identificación de marcadores cancerígenos con 92\% precisión).

    \item \textbf{Computación en la Nube para IA}
    
      \textit{AWS SageMaker}
    
      Plataforma que optimiza distribuciones paralelas de entrenamiento de modelos usando MapReduce y contenedores Docker.

    \item \textbf{Criptografía Cuántica}
    
      \textit{Algoritmo RSA mejorado}
    
      Implementaciones poscuánticas que resisten ataques basados en la factorización de enteros grandes (usado en blockchain).

    \item \textbf{Minería de Datos a Gran Escala}
    
      \textit{Algoritmo Apriori para Market Basket Analysis}
    
      Walmart lo utiliza para descubrir asociaciones entre productos (ej. "los que compran pañales tienen 65\% de probabilidad de comprar cerveza").

    \item \textbf{Simulaciones Computacionales}
    
      \textit{Modelado de Pandemias con SEIR}
    
      COVID-19 Forecast Hub combinó modelos diferenciales y Monte Carlo para predecir propagación con ±8\% error.
  \end{enumerate}

  \subsection*{Ventajas y Limitaciones}

  \textbf{Ventajas:}
  \begin{itemize}
    \item Precisión cuantificable mediante métricas matemáticas
    \item Escalabilidad en infraestructuras distribuidas
    \item Capacidad para manejar problemas de optimización pura
    \item Resultados reproducibles y verificables
  \end{itemize}

  \textbf{Limitaciones:}
  \begin{itemize}
    \item Dependencia crítica de la calidad de datos de entrada
    \item Costos energéticos elevados en cálculos masivos
    \item Rigidez ante problemas con restricciones dinámicas
    \item Riesgo de sesgos algorítmicos en modelos estadísticos
  \end{itemize}

  \textbf{Aplicación Actual}: Finanzas cuantitativas, predicción climática, diseño de fármacos mediante docking molecular, y optimización de cadenas de suministro globales.

  \subsection*{Ejemplos Adicionales}

  \begin{enumerate}
    \item \textbf{Predicción de Fallos en Infraestructuras con Modelos de Monte Carlo}

      \begin{itemize}
        \item \textbf{Problema}: Estimación probabilística de la vida útil de puentes bajo cargas dinámicas y corrosión.
        
        \item \textbf{Aplicación del paradigma}:

          \begin{itemize}
            \item Simulación de 10,000 escenarios mediante cadenas de Markov Monte Carlo (MCMC), integrando variables físicas (tensión, humedad) y ecuaciones diferenciales estocásticas.

            \item Uso de métodos numéricos para resolver integrales multidimensionales no analíticas.
          \end{itemize}

        \item \textbf{Beneficios del paradigma}:
        
          \begin{itemize}
            \item Capacidad para cuantificar riesgos con intervalos de confianza (ej. "Probabilidad del 95\% de fallo antes de 2035 ± 1.2 años").

            \item Optimización de recursos mediante algoritmos de muestreo adaptativo (reducción del 70\% en tiempo de simulación vs. métodos deterministas).
          \end{itemize}
      \end{itemize}

    \item \textbf{Personalización de Terapias Oncológicas mediante Optimización Convexa}

      \begin{itemize}
        \item \textbf{Problema}: Diseñar dosis de radiación que maximicen daño a tumores minimizando toxicidad en tejidos sanos.
        
        \item \textbf{Aplicación del paradigma}:

          \begin{itemize}
            \item Modelado matemático como problema de optimización restringida: función objetivo no lineal con 500+ variables (voxeles tumorales).
            
            \item Resolución mediante algoritmos de punto interior (IPM) con aceleración GPU para matrices dispersas.
          \end{itemize}
          
        \item \textbf{Beneficios del paradigma}:

          \begin{itemize}
            \item Precisión submilimétrica garantizada por convergencia matemática (error < 0.01\% en dosis objetivo).
            
            \item Escalabilidad a geometrías complejas (tumores irregulares) mediante descomposición de dominio.
          \end{itemize}
      \end{itemize}

    \item \textbf{Gestión de Tráfico Aéreo mediante Teoría de Juegos Algorítmica}

      \begin{itemize}
        \item \textbf{Problema}: Coordinar rutas de aviones comerciales para evitar colisiones y retrasos en espacio aéreo congestionado.

        \item \textbf{Aplicación del paradigma}:

          \begin{itemize}
            \item Formalización como juego cooperativo de n-jugadores con restricciones temporales y espaciales.

            \item Solución mediante algoritmos de equilibrio de Nash computacional, usando programación cuadrática secuencial (SQP).
          \end{itemize}

        \item \textbf{Beneficios del paradigma}:

          \begin{itemize}
            \item Garantías formales de seguridad mediante demostraciones matemáticas (ej. distancia mínima entre aviones siempre ≥ 5 km).

            \item Eficiencia computacional demostrable: complejidad \(O(n \log n)\) vs \(O(n^3)\) de métodos heurísticos.
          \end{itemize}
      \end{itemize}
  \end{enumerate}
\end{document}
